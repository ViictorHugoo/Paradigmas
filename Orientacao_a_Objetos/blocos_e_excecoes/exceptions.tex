\section{Exceções}

\begin{frame}[fragile]{Tratamento padrão de erros}

    \begin{itemize}
        \item O mecanismo padrão de SmallTalk-80 para sinalizar erros é o método
            \code{smalltalk}{error}

        \item Ele recebe como argumento a mensagem que será exibida caso o erro ocorra

        \item Se invocado, ele finalizará a execução do programa imediatamente

        \item Uma série de mensagens de erros, iniciada pela mensagem passada como parâmetros,
            indicarão os possíveis erros

        \item Algumas informações nestas mensagens podem ser utilizadas por ferramentas de
            diagnóstico

        \item Contudo, às vezes o erro a ser sinalizado por não ser fatal, permitindo que a 
            execução do programa continue

        \item Nestas situações, SmallTalk permite o uso de exceções
    \end{itemize}

\end{frame}

\begin{frame}[fragile]{Exceções}

    \begin{itemize}
        \item Em SmallTalk, todas as exceções são subclasses da classe \code{smalltalk}{Exception}

        \item O método \code{smalltalk}{printHierarchy} da classe \code{smalltalk}{Exception} lista
            todas as exceções já definidas pela linguagem

        \item Para tratar uma exceção, as mensagens \code{smalltalk}{on} e 
            \code{smalltalk}{do} devem ser passadas para o bloco que gera a exceção

        \item O primeiro indica a exceção que será tratada

        \item O segundo recebe como parâmetro o bloco de comandos que tratará a exceção

        \item Há seis ações básicas para o tratamento de exceções, todas elas passadas como
            mensagens para a instância da exceção
    \end{itemize}

\end{frame}

\begin{frame}[fragile]{Exemplo de tratamento de exceções em SmallTalk}
    \inputsnippet{smalltalk}{1}{21}{codes/division_by_zero.st}
\end{frame}


\begin{frame}[fragile]{Ações básicas de tratamento de exceções}

    \begin{itemize}
        \item \code{smalltalk}{return}: encerra o bloco passado para o método \code{smalltalk}{do},
            retornando o valor indicado. Se o valor for omitido, o retorno será igual a 
            \code{smalltalk}{nil}

        \item \code{smalltalk}{retry}: reinicia a execução do bloco que gerou a exceção. Deve ser
            usado com cuidado, pois pode gerar um laço infinito

        \item \code{smalltalk}{retryUsing}: reinicia a execução usando o bloco indicado, ao
            invés do bloco original

        \item \code{smalltalk}{pass}: repassa a exceção para o objeto que está acima do objeto
            original na hierarquia. Equivale ao \texttt{rethrow} de outras linguagens

        \item As outras duas ações básicas são \code{smalltalk}{resume} e \code{smalltalk}{outer}

        \item Se nenhuma das seis forem utilizadas, SmallTalk assumirá que o tratamento será
            \code{smalltalk}{sig return}

        \item Contudo, o indicado é que o tratamento seja dado explicitamente

        \item Para tratar mais de uma exceção com o mesmo bloco, separe com vírgulas os argumentos de
            \code{smalltalk}{on}
     \end{itemize}

\end{frame}

\begin{frame}[fragile]{Exemplo de tratamento de execeções}
    \inputsnippet{smalltalk}{1}{21}{codes/inverses.st}
\end{frame}
