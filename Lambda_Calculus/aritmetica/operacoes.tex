\section{Adição e multiplicação}

\begin{frame}[fragile]{Adição}

    \begin{block}{Adição de números naturais}
        Seja $S$ a expressão-$\lambda$ que computa o sucessor de um número natural. O 
        termo-$\lambda$
        \[
            + \equiv (\lambda xy.xSy)
        \]
        corresponde à \textbf{adição} de números naturais.
    \end{block}

    \vspace{0.1in}

    \textbf{Observação}: de acordo com a definição acima, a adição ($+$) é uma operação
    pré-fixada.
\end{frame}

\begin{frame}[fragile]{Exemplo de adição}

    \begin{align*}
        +23 &\equiv (\lambda xy.xSy)23 \\
        &\equiv 2S3 \\
        &\equiv S(S(3)) \\
        &\equiv S(4) \\
        &\equiv 5
    \end{align*}

\end{frame}

\begin{frame}[fragile]{Multiplicação}

    \begin{block}{Multiplicação de números naturais}
        A expressão-$\lambda$
        \[
            \times \equiv (\lambda xyz.x(yz))
        \]
        corresponde à \textbf{multiplicação} de números naturais.
    \end{block}

    \vspace{0.1in}

    \textbf{Observações}:
    \begin{enumerate}[(a)]
        \item Do mesmo modo que foi observado na adição, a multiplicação ($\times$) é uma operação 
            pré-fixada
        \item A interpretação desta expressão é a seguinte: $x$ marca o número de vezes que
            será aplicada a função $(yz)$, a qual aplica $y$ vezes a função $z$
    \end{enumerate}
\end{frame}

\begin{frame}[fragile]{Exemplo de multiplicação}

    \begin{align*}
        \times 23 &\equiv (\lambda xys.x(ys))23 \\
        &\equiv \lambda s.2(3s) \\
        &\equiv \lambda s.(\lambda yz.y(y(z))(3s) \\
        &\equiv \lambda s.(\lambda z.3s(3sz)) \\
        &\equiv \lambda s.(\lambda z.s(s(s(3sz))) \\
        &\equiv \lambda sz.s(s(s(s(s(s(z)))))) \\
        &\equiv 6
    \end{align*}

\end{frame}
